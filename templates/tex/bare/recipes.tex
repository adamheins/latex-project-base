\documentclass{article}

\usepackage[utf8]{inputenc}
\usepackage{parskip}
\usepackage{amssymb,amsfonts,amsmath,amscd}
\usepackage{bm}
\usepackage{hyperref}
\usepackage[pdftex]{graphicx}
\usepackage{url}
\usepackage[usenames,dvipsnames]{color}
\usepackage{enumitem}
\usepackage{mathtools}
\usepackage{float}

% Table stuff.
\usepackage{booktabs}
\usepackage{makecell}
\usepackage{multirow}

% biblatex for bibliography
\usepackage[
backend=biber,
maxcitenames=1,
style=authoryear,
]{biblatex}

\addbibresource{bibliography.bib}

% Useful commands.

% makecell config
\renewcommand\theadgape{}
\renewcommand\theadfont{\normalfont}

% colorize text
\newcommand\blue[1]{{\color{blue}#1}}
\newcommand\red[1]{{\color{red}#1}}

% math
\DeclareMathOperator*{\argmax}{argmax}
\DeclareMathOperator*{\argmin}{argmin}

\newcommand{\R}{\mathbb{R}}
\newcommand{\E}{\mathbb{E}}
\newcommand{\N}{\mathcal{N}}
\newcommand{\D}{\mathcal{D}}
\newcommand{\ident}{\mathbb{I}}
\newcommand{\var}{\text{var}}
\newcommand{\cov}{\text{cov}}
\newcommand{\pd}[2]{\frac{\partial #1}{\partial #2}}
\newcommand{\intinf}{\int_{-\infty}^{\infty}}


\title{This is the Title}
\author{Your Name}

\begin{document}

\maketitle

% Uncomment if you want to include a table of contents.
% \tableofcontents

% Uncomment if you want to include an abstract.
% \begin{abstract}
%   \input{abstract}
% \end{abstract}

\section{A Section}

% Recipe to include a figure.
% \begin{figure}[t]
%   \centering
%     \includegraphics[scale=1, trim = 0mm 0mm 0mm 0mm]{fig_name.ext}
%   \caption{Describe the figure.}
%   \label{fig:fig_name}
% \end{figure}

% Recipe for a table.
% \begin{table}[h]
%   \caption{Describe the table}
%   \centering
%     \begin{tabular}{l c c}
%       \toprule
%       Header 1 & Header 2 \\
%       \midrule
%       foo & bar \\
%       baz & bax \\
%       \bottomrule
%     \end{tabular}
%   \label{tab:table_name}
% \end{table}

% Recipe for splitting one line of a multiline equation
% \begin{equation}
%   \begin{alignedat}{2}
%     \log p(X\mid Y) &= && \log p(Y\mid X) + \log p(X_{out}) + \log p(x_0) \\
%                     &= && -\frac{d}{2}\log|K_y|-\frac{1}{2}(y-m_g(X))^TK_y^T(y-m_g(X))-\frac{1}{2}x_0^Tx_0 \\
%                     &  && -\frac{d}{2}\log|K_X|-\frac{1}{2}(X_{out}-m_f(X_{in}))^TK_X^{-1}(X_{out}-m_f(X_{in})) \\
%   \end{alignedat}
% \end{equation}

% Recipe for cases
% \begin{equation}
%   f(x) = \begin{cases}
%     1 & \text{if } x > 0 \\
%     0 & \text{else}
%   \end{cases}
% \end{equation}

% Recipe for multiple equations on the same line
\begin{align}
  a &= 0 & b &= 1
\end{align}

% Recipe for optimization
\begin{equation}
\begin{aligned}
  f(x^*) = \min_x      & \quad f_0(x) \\
           \text{s.t.} & \quad f_i(x) \leq b_i, \; i = 1, \ldots, m.
\end{aligned}
\end{equation}

% Recipe for under/overbrace
\begin{equation}
  y = \underbrace{a + b}_{\text{good}} + \overbrace{c + d}^{\text{bad}}
\end{equation}


% Uncomment for a bibliography.
% \newpage
% \printbibliography

% Uncomment for appendices.
% \newpage
% \appendix
% \input{appendix}

\end{document}
